\documentclass[11pt]{article}
\usepackage[a4paper,left=16mm, right=16mm, top=20mm, bottom=20mm]{geometry}
\usepackage[english]{babel}
\usepackage{hyperref}
\usepackage{multicol}
\usepackage{graphicx}
\usepackage{listings} 
\usepackage{color}

\definecolor{dkgreen}{rgb}{0,0.6,0}
\definecolor{gray}{rgb}{0.5,0.5,0.5}
\definecolor{mauve}{rgb}{0.58,0,0.82}

\def\sectionheader#1{\section*{#1}\vskip -0.3cm\hrule\vskip 0.3cm}

\def\image#1#2#3#4
{
    \begin{figure}[ht!]
        \centering
        \includegraphics[width=#4]{#1}
        \caption{#2 \label{#3}}
    \end{figure}
}

\newcommand\fnurl[2]{%
\href{#2}{#1}\footnote{\url{#2}}%
}

\title{Udacity Machine Learning Nanodegree \\ Capstone Project}
\author{Carsten Kr\"uger}
\date{\today}

\begin{document}

\lstset{frame=tb,
  language=Python,
  aboveskip=3mm,
  belowskip=3mm,
  showstringspaces=false,
  columns=flexible,
  basicstyle={\small\ttfamily},
  numbers=none,
  numberstyle=\tiny\color{gray},
  keywordstyle=\color{blue},
  commentstyle=\color{dkgreen},
  stringstyle=\color{mauve},
  breaklines=true,
  breakatwhitespace=true,
  tabsize=3
}

\maketitle

\section{Definition}

\subsection{Project Overview}

Machine learning is used in a wide variety of fields today. 
Luca Talenti et al. \cite{malaria} for example used a classification model to predict 
the severity criteria in imported malaria. In this project, machine learning will be used
to build a model that can decide based on the role information of an employee whether
that employee shall have access to a specific resource. 
\\ \\
An employee that has to use a computer in order to fulfill their tasks, needs access 
to certain areas of software programs or access rights to execute actions such as read, 
write or delete a document. While working, employees may encounter that they don't have a 
concrete access right required to perform the task at hand. In those situations a supervisor or an 
administrator has to grant them access. The process of discovering that a certain access 
right is missing and removing that obstacle is both time-consuming and costly.
A model that can predict which access rights are needed based on the current role of an 
employee is therefore relevant.
    
\subsection{Problem Statement}

The problem stems from the {\it Amazon.com Employee Access Challenge Kaggle Competition} 
\cite{kaggleAmazon} and is there described as follows:

{\it ``The objective of this competition is to build a model, learned using 
historical data, that
 will determine an employee's access needs, such that manual access transactions 
 (grants and revokes) are minimized as the employee's attributes change over time. 
 The model will take an employee's role information and a resource code and will return whether 
 or not access should be granted.''}
\\ \\
This is a supervised learning problem because the dataset is labeled. 
Anticipated solution:

\begin{enumerate}
    \item Explore data in order to gain insights.
    \item Train many different binary classification models using standard parameters.
    \item Apply transformations or regularizations.
    \item Compare plain models and transformed models.
    \item Pick the three best models based on the performance metric.
    \item Tweak the chosen models in order to improve model performance.
    \item Evaluate the tweaked models on the test set.
    \item Conclusion
\end{enumerate}

\subsection{Metrics}

To quantify model performance, the area under the ROC curve will be used.
This metric is appropriate for this type of project because it works well
even if the classes are not balanced. Moreover it was the metric of choice
in the herein before mentioned Kaggle competition.
The metric is derived by first constructing the ROC curve and then 
calculating the area under that curve. 

{\it``The ROC curve is created by plotting the true positive rate against the false 
positive rate at various threshold settings''} \cite{rocCurve},
 
where the threshold is a value between 0 and 1 that determines how sure the model needs 
to be in order to classify a data entry as positive (access granted in the problem at hand). 
For example if the threshold was 0.7 the model would have to have calculated a probability 
of at least 70 $\%$ to classify a data entry as positive.

\section{Analysis}

\subsection{Data Exploration}

There are $32769$ entries in the dataset with no missing values. 
Figure \ref{first_5} shows the first five rows in the dataset.

\image{images/dataset_head.png}{Top five rows in the dataset}{first_5}{140mm}

The dataset has ten attributes. All attributes are categorial. 
One attribute called {\tt RESOURCE} holds the ID of the resource 
for which the access has been granted or denied. 
There are $7518$ different resources in the dataset.
The target attribute is called {\tt ACTION}.
The other eight columns provide role information 
for an \fnurl{employee}{https://www.kaggle.com/c/amazon-employee-access-challenge/data}:

\begin{itemize}
    \item {\tt MGR\_ID} - ID of the manager of employee ($4243$ different values)
    \item {\tt ROLE\_ROLLUP\_1} - Role ID of employee ($128$ different values)
    \item {\tt ROLE\_ROLLUP\_2} - Second role ID of employee ($177$ different values)
    \item {\tt ROLE\_DEPTNAME} - Role department description ($449$ different values)
    \item {\tt ROLE\_TITLE} - Role business title ($343$ different values)
    \item {\tt ROLE\_FAMILY} -  Role family description ($67$ different values)
    \item {\tt ROLE\_FAMILY\_DESC} - Extended role family description ($2358$ different values)
    \item {\tt ROLE\_CODE} - Company role code; this code is unique to each role ($343$ different values)
\end{itemize}
\noindent
Because the eight attributes that describe the role of an employee and the resource
attribute are categorial, they will be vectorized. 
This is achieved by using one-hot encoding. For example the {\tt ROLE\_FAMILY}-attribute
has $67$ different categories. Each row in the dataset that has been one-hot encoded 
will contain one entry for each category. The entry will be $1$ (hot) only for the 
entry corresponding to the current role family  and $0$ (cold) for the 
$66$ other role families. This example shows that the number of input attributes will
increase tremendously.

\subsection{Exploratory Visualization}

Figure \ref{histogram} shows that the {\tt ACTION}-attribute is highly unbalanced. 
In fact more than $94 \%$ of the rows have an {\tt ACTION}-attribute 
of $1$ (access granted) whereas only roughly $6 \%$ have a $0$ (access denied).
The accuracy metric is therefore inadequate for this dataset because even
a dumb model that always predicts $1$ would have a very high accuracy.

\image{images/action_histogram.png}{Histogram for target attribute}{histogram}{120mm}

\subsection{Algorithms and Techniques}

First of all it is important to put aside a test set. This set will not be touched until the very end
when all models have been trained and optimized. To generate the test set
\fnurl{stratified shuffle split}{http://scikit-learn.org/stable/modules/generated/sklearn.model_selection.StratifiedShuffleSplit.html}  
will be used which preserves the percentage of samples for each class.
A preprocessing pipeline will be created, that selects role attributes and applies
one-hot encoding to them. This step is important because, as mentioned before,
the attributes are categorial and two values that are close to each other are not
more similar than two values with a larger distance.
Different binary classification models with standard parameters will be created.
The classifiers trained in this project are Logistic Regression, 
Decision Tree, SVM, Random Forest, AdaBoost and XGBoost.
They will be initialized with a fixed random seed in order to make the results reproducable.
There performance will be measured using scikit learn's 
\fnurl{cross validation score}{http://scikit-learn.org/stable/modules/generated/sklearn.model_selection.cross_val_score.html} 
function in order to circumvent overfitting. 
The best models will be tweaked by trying different sets of parameters.
The goal is to furhter increase their performance. The better set of hyperparameters
for the different models will be determined by doing a
\fnurl{grid search}{http://scikit-learn.org/stable/modules/generated/sklearn.model_selection.GridSearchCV.html}.
Grid search will evaluate all combinations of a specified amount of parameters and values.
This is far more convenient than the tedious task of trying out a bunch of hyperparameters manually.
Finally the previously generated test set will be used to determine
whether the final model generalizes well.

\subsection{Benchmark}

An out of the box logistic regression model will be used as the benchmark for 
this project, because the model is fast, simple to implement and to interpret 
and should give far better results than random guessing for the problem at hand.
\\ \\
Because the problems stems from a Kaggle competition, as a secondary benchmark, 
the result of the final solution will be compared to the result of the solution 
of the team that won the competition. 
The submissions to the competition were judged on the 
{\it area under the ROC curve (auc)} metric. 
Therefore this metric will be used to compare the results. 
The winning team got an {\it auc} value 
of $0.92360$, which is an excellent result.

\section{Methodology}

\subsection{Data Preprocessing}

Compared to the samples with a positive
class (access granted) the dataset contains very few samples with a negative class
(access denied). The first preprocessing step is therefore to put aside a test set 
that preserves the percentage of samples for each class. As mentioned before this is 
achieved by using a stratified shuffle split.
If a standard random split had been used to
split the dataset into a training and testing set, it would be possible that
for example the training set would not contain a single negative sample which
would have certainly a negative impact on the performance of the classifiers.
The second preprocessing step is to one-hot encode the predictive attributes.
This is done by using sklearn's
\fnurl{{\it OneHotEncoder}}{http://scikit-learn.org/stable/modules/generated/sklearn.preprocessing.OneHotEncoder.html}.
This transforms all the categorial attributes to binary attributes and fixes the
issue that the models would assume that two nearby values of a categorial
attribute are more similar than two distant values.

\subsection{Implementation}

The first step is to make some imports and to read the dataset:
\begin{lstlisting}[frame=single]
    import numpy as np
    import pandas as pd
    random_state = 42
    df = pd.read_csv('../data/train.csv')
\end{lstlisting}
\noindent  
The second step is to create the stratified training and test sets with a test size of $25 \%$:

\begin{lstlisting}[frame=single]
    from sklearn.model_selection import StratifiedShuffleSplit
    sss = StratifiedShuffleSplit(n_splits=1, test_size=0.25, random_state=random_state)
    for train_index, test_index in sss.split(df, df['ACTION']):
        train_set, test_set = df.loc[train_index], df.loc[test_index]
\end{lstlisting}
\noindent
The third step is to extract the labels and the attributes for the training set:

\begin{lstlisting}[frame=single]
    access = train_set.drop('ACTION', axis=1)
    access_labels = train_set['ACTION'].copy()
\end{lstlisting}
\noindent
The fourth step uses a {\tt DataFrameSelector}:

\begin{lstlisting}[frame=single]
    # DataFrameSelector class, taken from "Hands-On Machine Learning with Scikit-Learn & 
    # Tensorflow by Aurélion Géron (O'Reilly). Copyright 2017 Aurélion Géron, 978-1-491-96229-9, Page 41
    from sklearn.base import BaseEstimator, TransformerMixin
    class DataFrameSelector(BaseEstimator, TransformerMixin):
        def __init__(self, attribute_names):
            self.attribute_names = attribute_names
        def fit(self, X, y=None):
            return self
        def transform(self, X):
            return X[self.attribute_names].values
\end{lstlisting}
\noindent
Using the {\tt DataFrameSelector} a pipeline is built that one-hot encodes the attributes of the 
training set:

\begin{lstlisting}[frame=single]
    # get attributes
    attributes = access.columns.values.tolist()

    from sklearn.pipeline import Pipeline
    from sklearn.preprocessing import OneHotEncoder

    # one-hot encode the categorial attributes
    pipeline = Pipeline([
        ('selector', DataFrameSelector(attributes)),
        ('encoder', OneHotEncoder())
    ])
    access_1hot = pipeline.fit_transform(access)
\end{lstlisting}
\noindent
The sixth step is to create a function that takes a trained model and an integer
number of folds as an input and displays the cross validation scores based
on the area under the ROC curve metric ({\tt auc}) for the $n$-folds,
their mean and their standard deviation:

\begin{lstlisting}[frame=single]
    def display_auc_scores(model, folds=10):
        scores = cross_val_score(model, access_1hot, access_labels, scoring='roc_auc', cv=folds)
        print('Scores: ', scores)
        print('Mean: ', scores.mean())
        print('Std: ', scores.std())
\end{lstlisting}
\noindent
The final step is to actually create and train the different models. 
The following listing shows the creation and training 
of a logistic regression classifier:

\begin{lstlisting}[frame=single]
    from sklearn.linear_model import LogisticRegression
    log_reg = LogisticRegression(random_state=random_state)
    log_reg.fit(access_1hot, access_labels)
    display_auc_scores(log_reg, folds=10)
\end{lstlisting}

\subsection{Refinement}
The initial results for the {\bf logistic regression} benchmark model on the training-set
were the following:
\begin{lstlisting}[frame=single]
    ('Scores: ', array([ 0.86783489,  0.79915939,  0.79383764,  0.87851276,  0.85076963,
        0.86988866,  0.86555532,  0.85809023,  0.84325891,  0.86433091]))
    ('Mean: ', 0.84912383267572067)
    ('Std: ', 0.027958452541170693)
\end{lstlisting}
\noindent
The initial results for all models ordered by descending mean {\it auc} score 
on the training-set were the following:

\begin{center}
    \begin{tabular}{l|r|r}
        {\bf Model} & {\bf Mean auc} & {\bf Standard Deviation} \\ \hline 
        Logistic Regression & $0.84912$ & $0.02796$ \\
        Random Forest & $0.82307$ & $0.02451$ \\
        XGBoost & $0.75264$ & $0.02680$ \\
        SVM & $0.75006$ & $0.03236$ \\
        AdaBoost & $0.74251$ & $0.02921$ \\
        Decision Tree & $0.70193$ & $0.02216$ \\
    \end{tabular}
\end{center}
\noindent
\\ \\
To refine the initial results the best three models were chosen to being tweaked
with the exception of the logistic regression model that was used as the benchmark.
For each hyperparameter that has been modified the default value was the basis. 
Values that are near this default value were tested as well as some values
that are farther away. In the end the best parameters according to the 
{\it auc} scores obtained by doing a grid search cross validation were chosen.
\\ \\
It was first tried to improve the random forest model
by discovering a good value for the {\tt n\_estimators}-parameter
corresponding to the number of trees in the forest. The
default value is $10$ and values of $5, 10, 20, 30, 40$ and $50$ have been tested.
The best number of trees was discovered to be $50$. With this parameter-change
the model's mean {\it auc} score on the training-set increased to $0.85387$.
Secondly the number of minimum required samples to split an internal node has been
examined. The default value is $2$ and values of $2, 5$ and $10$ have been tested.
The best value has discovered to be $5$. With this additional change the
mean {\it auc} score on the training-set increased to $0.85687$. By the setting the
max depth to $500$ the mean {\it auc} score got up to $0.85706$. Changing
the {\tt min\_samples\_leaf} parameter from the default
value of $1$ to $2, 3, 4, 10, 50, 100$ brought no improvement. Finally the 
maximum number of features to consider when looking for the best split has been
tried to improve. The values {\it sqrt}, $5, 10, 20$ and $50$ have been tried.
Among those values, $10$ has been discovered to be the best and with that
the mean {\it} auc score on the training-set increased to $0.86371$.
\\ \\
Secondly the SVM model has been improved by doing a grid-search 
with the values $[0.01, 0.1, 1, 10]$ for the
parameter {\tt gamma} and $[0.1, 1, 10, 100]$ for the parameter {\tt C}.
The best values were $1$ for {\tt gamma} and $10$ for {\tt C}. This search
took several hours. Even the score calculation of the refined SVM model
took a long time. Because of the extremely high computational costs, 
other parameter settings for the SVM model were not tried.
The refined SVM got a mean {\it auc} score on the training-set of $0.85768$.
\\ \\
The last model being refined was the XGBoost model. The first grid-search
consisted of the parameters {\tt max\_depth} and {\tt min\_child\_width}
for which the values $[5, 6, 7, 8, 9, 10]$ and $[1, 2, 3, 4, 5]$ were used respectively.
Out of these parameter-sets the best values were $10$ for the max depth
and $1$ for the other parameter. With this parameter-setting the 
mean {\it auc} score on the training-set inccreased to $0.81301$.
Trying other values such as $[6, 7, 8, 9, 10, 50, 100$ for the {\tt min\_child\_width}
parameter did not lead to better results. For the {\tt gamma} parameter the
value $0.5$ was found to give the best results on the training set. 
The default value is $0.0$ and values of $0.0, 0.1, 0.2, 0.3, 0.4, 0.5, 0.6$ and $1.0$
have been tried. After this parameter twist has been applied the score got up to
$0.81489$. The best learning rate was found to be $0.4$. With this little change
the score increased to $0.82793$. Next, I've tried to find a good value for the
number of boosted trees to fit. After trying many different values I discovered
that the best value seems to be $170$. Using this value for the parameter led
to a mean score of $0.83418$. After that I've tried different values for
the subsample ratio of the training instance and the subsample ratio of columns when
each each tree is constructed. This are the parameters {\tt subsample} and
{\tt colsample\_bytree}. The best value for the {\tt subsample} parameter
seems to be the default value of $1.0$. For the {\tt colsample\_bytree} parameter
I found a value of $0.9$ to be better. With this adjustment the score
increased to $0.83470$. I was not able to find better values for the regularization
parameters {\tt reg\_alpha} and {\tt reg\_lambda} than the defaults. The same holds 
true for the parameter that is used for balancing of positive and negative weights.
After that I again tried some values for the number of boosted trees and discovered
that much higher numbers led to somewhat improved scores on the training set.
With a value of $1000$ the score increased to $0.83897$. But the change comes with
a cost the computation time increased perceptibly. Therefore higher numbers
were not tested for this parameter.
\section{Results}

\subsection{Model Evaluation and Validation}

To evaluate the model the previously set aside test-set was used. 
The parameters which had different values than the default value
of the XGBoost model, that was found to be the strongest
among those tested, were the following:

\begin{center}
    \begin{tabular}{l|r}
        {\bf Parameter} & {\bf Value} \\ \hline 
        Max Depth & $10$  \\
        Min Child Weight & $1$  \\
        Gamma & $0.5$  \\
        Learning Rate & $0.4$  \\
        \# of Estimators & $1000$  \\
    \end{tabular}
\end{center}
\noindent
The parameter-values of this final model were found using grid-search with many
different values. They gave the best results on the training-set. Cross-validation
with $10$ folds was used to train the hyperparameters and the mean {\it auc}
score of the final model was very promissing, giving a score of about $0.83897$.
Unfortunately though the model does not seem to generalize very well 
because the {\it auc} score on the test-set is rather low. 
This indicates that the model is not as robust as desired and results found
by the model cannot be trusted.

\subsection{Justification}

On the test-set the benchmark logistic regression model got an {\it auc} score of $0.56946$.
The three models that have been used for further examination were all
found slightly stronger than the benchmark model. The final 
random forest model had a score of $0.57810$, the final SVM model a score of
$0.57243$ and the final XGBoost model got a score of $0.65055$. 
Figure \ref{roccurves} shows the ROC curves of the final model compared to the benchmark.

\image{images/roc_curve.png}{ROC curves for different models}{roccurves}{120mm}
\noindent
So the XGBoost model is the one that can handle unseen data better than the other models.
The overall score is not that good. As mentioned before the winning team of the
Kaggle competition managed to get a score of $0.92360$. Because of that
I would argue that the final solution is not significant enough to have 
solved the problem. At least it is better than the benchmark and far better than
random guessing.
\newpage
\section{Conclusion}

\subsection{Free-Form Visualization}

Figure \ref{learningcurves} shows the performance
of the final XGBoost model on the training and validation set 
as a function on the training-set size.

\image{images/learning_curves.png}{Learning curves of final XGBoost model}{learningcurves}{120mm}
\noindent
It is evident that the model is overfitting because there is a clear gap
between the training scores and the validation scores, where the training
scores are much higher.

\subsection{Reflection}

The overall process can be summarized as follows:

\begin{itemize}
    \item Analysis of the problem and data
    \item Splitting data into distinct training- and testing-sets
    \item Preprocess the data, i.e. one-hot-encode the categorial features
    \item Training different models on the training-set
    \item Picking the three models with the best performance on the training set
    \item Try to improve the best models by modifying the hyperparameters of the
    best three models
    \item Apply the preprocessing step to the test-set
    \item Evaluate the performance of the models on the test-set
    \item Compre the final models with the benchmark
\end{itemize}
\noindent
One of the biggest problems or challenges of this project was that training
and evaluating the models took a lot of time. Especially the SVM model is
very computational expensive. Because of the high-dimensionality of the problem,
having more than $14,000$ features, it is not possible to draw decision
boundaries and therefore not possible to get a mental image of the solution.
It was also hard to find a good solution that generalizes well and give
good results on both the training- and testing-sets. In the end I was not
able to achieve good {\it auc} scores on both sets. 
Nontheless it was interesting to work on this project
because it allowed me to get a hands on experience on a real world problem
and to apply some of my knowledge which I gained during this Udacity course.
\subsection{Improvement}

With a better computer that had more CPU power it would be possible to 
try a lot more different sets of parameters for the models.
For example the number of trees in the random forest model could probabaly 
be increased in order to improve the performance. 
Using different kernels for the SVM might be another possibility for improvement. 
A good approach might also be to reduce the dimensionality of the problem by 
performing a principle component analysis. This step could be integrated into
the preprocessing pipeline. Another aspect is that figure \ref{learningcurves} 
shows that the validation score goes up when the training size increases. 
The different models might therefore benefit from more training data. 

\begin{thebibliography}{9}

    \bibitem{malaria}
    L1 logistic regression as a feature selection step for training stable 
    classification trees for the prediction of severity criteria in imported malaria

    \textit{Luca Talenti, Margaux Luck, Anastasia Yartseva, Nicolas Argy, Sandrine Houzé, Cecilia Damon}

    \href{https://arxiv.org/abs/1511.06663}{arXiv:1511.06663 [cs.LG]}

    \bibitem{kaggleAmazon}
    Amazon.com -- Employee Access Challenge 

    \textit{Predict an employee's access needs, given his/her job role.}

    \href{https://www.kaggle.com/c/amazon-employee-access-challenge}
    {https://www.kaggle.com/c/amazon-employee-access-challenge},


    \bibitem{rocCurve}
    Receiver operating characteristic 

    \href{https://en.wikipedia.org/wiki/Receiver_operating_characteristic}
    {https://en.wikipedia.org/wiki/Receiver\_operating\_characteristic}

\end{thebibliography}

\end{document}
    
    
    