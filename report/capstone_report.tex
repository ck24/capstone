\documentclass[11pt]{article}
\usepackage[a4paper,left=16mm, right=16mm, top=15mm, bottom=20mm]{geometry}
\usepackage{hyperref}
\usepackage{multicol}

\def\sectionheader#1{\section*{#1}\vskip -0.3cm\hrule\vskip 0.3cm}

\title{Udacity Machine Learning Nanodegree \\ Capstone Project}
\author{Carsten Kr\"uger}
\date{\today}

\begin{document}

\maketitle

\section{Definition}

\subsection{Project Overview}

Machine learning is used in a wide variety of fields today. 
Luca Talenti et al. \cite{malaria} for example used a classification model to predict 
the severity criteria in imported malaria. In this project, machine learning will be used
to build a model that can decide based on the role information of an employee whether
that employee shall have access to a specific resource \cite{kaggleAmazon}. 
\\ \\
An employee that has to use a computer in order to fulfill their tasks, needs access 
to certain areas of software programs or access rights to execute actions such as read, 
write or delete a document. While working, employees may encounter that they don't have a 
concrete access right required to perform the task at hand. In those situations a supervisor or an 
administrator has to grant them access. The process of discovering that a certain access 
right is missing and removing that obstacle is both time-consuming and costly.
A model that can predict which access rights are needed based on the current role of an 
employee is therefore relevant.
    
\subsection{Problem Statement}

The problem stems from the {\it Amazon.com Employee Access Challenge Kaggle Competition} 
\cite{kaggleAmazon} and is there described as follows:

{\it ``The objective of this competition is to build a model, learned using historical data, that
 will determine an employee's access needs, such that manual access transactions 
 (grants and revokes) are minimized as the employee's attributes change over time. 
 The model will take an employee's role information and a resource code and will return whether 
 or not access should be granted.''}
\\ \\
This is a binary classification problem.

\subsection{Metrics}

\section{Analysis}

\subsection{Data Exploration}

\subsection{Exploratory Visualization}

\subsection{Algorithms and Techniques}

\subsection{Benchmark}

\section{Methodology}

\subsection{Data Preprocessing}

\subsection{Implementation}

\subsection{Refinement}

\section{Results}

\subsection{Model Evaluation and Validation}

\subsection{Justification}

\section{Conclusion}

\subsection{Free-Form Visualization}

\subsection{Reflection}

\subsection{Improvement}

\begin{thebibliography}{9}

    \bibitem{malaria}
    L1 logistic regression as a feature selection step for training stable 
    classification trees for the prediction of severity criteria in imported malaria

    \textit{Luca Talenti, Margaux Luck, Anastasia Yartseva, Nicolas Argy, Sandrine Houzé, Cecilia Damon}

    \href{https://arxiv.org/abs/1511.06663}{arXiv:1511.06663 [cs.LG]}

    \bibitem{kaggleAmazon}
    Amazon.com -- Employee Access Challenge 

    \textit{Predict an employee's access needs, given his/her job role.}

    \href{https://www.kaggle.com/c/amazon-employee-access-challenge}
    {https://www.kaggle.com/c/amazon-employee-access-challenge},


    \bibitem{rocCurve}
    Receiver operating characteristic 

    \href{https://en.wikipedia.org/wiki/Receiver_operating_characteristic}
    {https://en.wikipedia.org/wiki/Receiver\_operating\_characteristic}

\end{thebibliography}

\end{document}
    
    
    